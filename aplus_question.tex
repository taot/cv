\documentclass[a4paper]{article}
\usepackage[utf8]{inputenc}
\usepackage[cm]{fullpage}
\input apl

\title{Question}
\date{}

\begin{document}
\maketitle

\section{What is the function of the following expression?}

\begin{verbatim}
((x⍳x)=⍳#x)/x
\end{verbatim}

\section{Operation Definitions}

Assume we have the following five operations defined. The data can be
either an integer or a vector (array) of integers.

\subsection{Find: y⍳x}

x and y are two vectors. The result is a vector of same length as
 x. The i'th element of result is index of the first item of y that is
 identical to i'th element of x; if a match is not found, the result
 item is the length of y. For example,

\begin{verbatim}
>   2 3 ⍳ 1 2 3 4 5
2 0 1 2 2
\end{verbatim}

\subsection{Interval: ⍳x}

Takes an integer and returns a vector of integers from 0 to
 x-1. For example,

\begin{verbatim}
>   ⍳5
0 1 2 3 4
\end{verbatim}

\subsection{Count: \#x}

Takes a vector and returns the number of items of x. For example,
\begin{verbatim}
>   #1 2
3 4 4
\end{verbatim}

\subsection{Equal to: x=y}

x and y are two vectors of same length. Returns a vector whose i'th
 element is 1 if x[i] (means i'th element in x) and y[i] are equal,
 otherwise 0. For example,
\begin{verbatim}
>   1 2 3 4 5 = 1 0 3 4 6
1 0 1 1 0
\end{verbatim}

\subsection{Replicate: y/x}

x and y are two vectors of same length. The result is a vector, whose
items are taken from the x. Each item in x (x[i]) is replicated y[i]
times in the result. For example,
\begin{verbatim}
>   1 2 3 / 4 5 6
4 5 5 6 6 6
\end{verbatim}

\end{document}
