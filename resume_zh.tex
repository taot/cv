\documentclass[10pt,a4paper]{moderncv}

% moderncv themes
\moderncvtheme[blue]{classic}

% adjust the page margins
\usepackage[scale=0.8]{geometry}

% use xeCJK and set Chinese font
\usepackage{xeCJK}

% the following works for Linux (Only outline fonts work)
\setCJKmainfont{WenQuanYi Zen Hei Sharp}
\setCJKfamilyfont{sf}{WenQuanYi Zen Hei Sharp}

% the following works for windows
%\setCJKmainfont{SimSun}
%\setCJKfamilyfont{sf}{SimSun}

% personal data
\firstname{陶}
\familyname{涛}
%\title{Resumé title (optional)}
\mobile{手机: 18217618487}
\email{电子邮件: 78019092@qq.com}

%----------------------------------------------------------------------------------
%            content
%----------------------------------------------------------------------------------
\begin{document}
\maketitle

\section{个人信息}
\cvdoubleitem{性别}{男}{学历}{硕士}
\cvdoubleitem{专业}{计算机科学与技术}{工作年限}{9年}

\section{自我评价}
\cvlistitem{精通 JAVA 及相关技术和框架。}
\cvlistitem{熟悉测试驱动和敏捷开发。}
\cvlistitem{良好的逻辑思维能力及抽象能力。}
\cvlistitem{能以英语熟练交流。}

\section{工作经验}
\cventry{2016.5至今}{软件部负责人}{青岛国数信息科技有限公司}{}{}{
  \begin{itemize}
  \item 负责软件架构设计
  \item 组建软件开发团队,并带领团队开发
  \item 设计开发无人机监控平台
  \item 设计开发北斗 RDSS 远程通信终端软件
\end{itemize}}
\cventry{2013.4--2016.4}{架构师}{通联数据有限公司(DataYes)}{}{}{
  \begin{itemize}
  \item 负责PMS组合管理系统的技术架构设计,带领团队进行开发。
  \item 设计基于Java, Scala, Play Framework, Javascript的系统框架,前后端使用Restful风格接口调用
  \item 与投资团队沟通需求,并协调和其他团队的合作。
  \item 从数据库和JVM等方面做性能优化
\end{itemize}}
\cventry{2012.6--2013.4}{高级软件工程师/架构师}{上海瑞医信息科技有限公司}{}{}{
  带领团队进行网站后台开发,进行用户数据后台收集。维护生产环境和测试环境的运行。
  \begin{itemize}
  \item 构建网站的用户信息跟踪系统,分析用户数据。
  \item 设计并实现网站的新用户系统,从旧用户系统迁移至新系统。
\end{itemize}}
\cventry{2011.5--2012.6}{高级软件工程师}{软通动力}{}{}{
  驻摩根士丹利(上海)软件服务公司。负责设计、实现、维护利率衍生品(Interest Rate Derivative)的收益曲线服务业务(Yield Curve Service),给trader,strats和其他tream提供服务和支持。
  \begin{itemize}
  \item 与trader,strats和其他team沟通,了解业务需求。
  \item 设计并实现intraday marking to market系统,并进行测试。
  \item 参与实现固定收益部门(Fixed Income)的下一代风险分析系统
\end{itemize}}

\cventry{2008.3--2011.4(海外)}{软件工程师}{Oracle}{}{}{
  负责公司主要产品(P6 Web Services, P6 Web Access和P6 Integration API等)的开发工作,参与团队的项目管理。分析用户需求,设计并实现新功能,维护已有功能,修正bug。改进产品性能使之更加适合企业级部署。发布产品的重大版本8.0,并且发布多个service pack以及hotfix版本。}

\section{教育情况}
\cventry{2006.8--2008.12}{硕士,计算机科学}{Villanova大学}{美国费城}{\textit{GPA: 3.8/4.0}}{}
\cventry{2002.9--2006.7}{学士,计算机科学与技术}{北京邮电大学}{北京}{}{}

\section{语言能力}
\cvlanguage{英语}{流利}{\hfill 在美国学习工作5年,能够用英语流利地进行口语和书面交流}

\end{document}
