\documentclass[11pt,a4paper]{moderncv}

% moderncv themes
\moderncvtheme[blue]{classic} 

% adjust the page margins
\usepackage[scale=0.8]{geometry}

% use xeCJK and set Chinese font
\usepackage{xeCJK}

% the following works for Linux (Only outline fonts work)
\setCJKmainfont{WenQuanYi Zen Hei Sharp}
\setCJKfamilyfont{sf}{WenQuanYi Zen Hei Sharp}

% the following works for windows
%\setCJKmainfont{SimSun}
%\setCJKfamilyfont{sf}{SimSun}


% personal data
\firstname{陶}
\familyname{涛}
%\title{Resumé title (optional)} 
\mobile{手机: 18217618487}
\email{电子邮件: librakevin@gmail.com}

%----------------------------------------------------------------------------------
%            content
%----------------------------------------------------------------------------------
\begin{document}
\maketitle

\section{个人信息}
\cvdoubleitem{性别}{男}{学历}{硕士}
\cvdoubleitem{专业}{计算机科学与技术}{工作年限}{4年}
%\cvdoubleitem{目前年薪}{20万}

%\section{求职意向}
%\cvline{工作性质}{全职}
%\cvline{希望行业}{计算机软件,金融/投资/证券}
%\cvline{目标地点}{杭州}
%\cvline{期望薪水}{面议}
%\cvline{目标职能}{高级软件工程师,软件工程师,系统分析员}
%\cvline{目前状况}{在美国工作,正在考虑回国发展}

\section{自我评价}
\cvlistitem{精通JAVA及相关技术,有很强的分析能力和解决问题能力。}
\cvlistitem{美国大学硕士毕业,并在美国有3年软件行业工作经验,能以英语熟练交流。}
\cvlistitem{熟悉测试驱动和敏捷开发,熟悉大型软件的开发发布流程。}
\cvlistitem{熟悉项目管理的业务流程,有一定的项目管理经验,有良好的沟通能力。}

%\section{专业技能}
%\cvcomputer{Java}{Ant, Eclipse, }{J2EE}{精通}
%\cvcomputer{Object-Oriented}{精通}{Linux}{精通}
%\cvcomputer{Oracle}{熟练}{SQL Server}{熟练}
%\cvcomputer{C/C++}{熟练}{}{}

\section{教育情况}
\cventry{2006.8--2008.12(海外)}{硕士,计算机科学}{Villanova大学}{美国费城}{\textit{GPA: 3.8/4.0}}{\hfill\newline
获学校全额奖学金资助。担任助研,协助教授的研究工作。\newline
参与XMPP协议的研究,并使用CPN Tools对XMPP协议进行建模。\newline
毕业设计用particle filter来实现机器人的定位(Robot Localization using Particle Filter)。\newline}
\cventry{2002.9--2006.7}{学士,计算机科学与技术}{北京邮电大学}{北京}{}{\hfill\newline
成绩优秀,多次获得奖学金。学院内排名Top 5\%。\newline
毕业设计为基于嵌入式Linux的视频监视系统,并获得毕业设计优秀奖。\newline}

\section{语言能力}
\cvlanguage{英语}{精通}{\hfill 在美国学习工作5年,能够用英语流利地进行口语和书面交流}


%\newpage


\section{工作经验}
\cventry{2011.5至今}{高级软件工程师}{软通动力}{}{}{
外派至摩根士丹利(上海)软件服务公司。负责设计、实现、维护利率衍生品(Interest Rate Derivative)的收益曲线服务业务(Yield Curve Service),给trader,strats和其他tream提供服务和支持。
\newline\newline
主要成就:
\begin{itemize}
\item 与trader,strats和其他team沟通,了解他们的需求。
\item 设计并实现intraday marking to market系统,并进行测试。
\item 参与实现固定收益部门(Fixed Income)的下一代风险分析系统
\end{itemize}}
\newline

\cventry{2010.3--2011.4(海外)}{软件工程师}{Oracle}{}{}{
负责公司主要产品(P6 Web Services, P6 Web Access和P6 Integration API等)的开发工作,参与团队的项目管理。分析用户需求,设计并实现新功能,维护已有功能,修正bug。改进产品性能使之更加适合企业级部署。发布产品的重大版本8.0,并且发布多个service pack以及hotfix版本。
\newline\newline
主要成就:
\begin{itemize}
\item 针对产品的性能进行调试,发现性能瓶颈,并对之进行优化,提升产品性能
  \begin{itemize}
  \item 监视数据库的SQL请求,寻找并优化性能低下的请求。
  \item 对用户界面(Swing)进行优化,提高用户的使用体验。
  \item 优化事件子系统的XML生成代码,成倍提高了事件子系统的性能。
  \end{itemize}
\item 领导P6产品的事件机制的开发工作。
  \begin{itemize}
  \item 为P6产品设计了基于JMS 1.1的事件机制。
  \item 设计了事件机制的代码架构。
  \item 参与编码实现,并指导其他开发人员进行编码。
  \item 设计并实现了自动测试工具对事件机制进行集成测试。
  \item 指导测试人员编写测试脚本。
  \end{itemize}
\item 曾担任scrum master,帮助项目经理管理项目和团队。
  \begin{itemize}
  \item 在每个scrum sprint开始时做计划,安排任务。
  \item 主持每天的scrum会议,了解任务进度并讨论存在的问题。
  \item 确保每个特性都有足够的测试脚本或者手动测试来保证其质量。
  \item 在sprint中期做health check,并根据情况调整项目计划和任务分配。
  \item 在公司的sprint review会议上报告项目进展。
  \end{itemize}
\item 曾于一个月之内发布5个版本的不同产品。
  \begin{itemize}
  \item 高强度下按期完成任务,并保证产品质量。
  \item 设计实现新特性,并且修正Bug。
  \item 完成各版本的集成测试,回归测试和checklist。
  \end{itemize}
\end{itemize}}
\newline

\cventry{2009.1--2010.2(海外)}{初级软件工程师}{Oracle}{}{}{
负责公司主要产品的开发工作,添加新功能并维护已有功能,修正bug。发布各个产品的重大版本7.0,并发布多个其他版本。
\newline\newline
主要成就:
\begin{itemize}
\item 参与开发合同管理软件Contract Management Web Services。
  \begin{itemize}
  \item 参与产品的架构设计(后台基于已有产品,前端使用Apache CXF构建web services,并大量使用代码生成来产生许多中间代码)。
  \item 率先在公司内使用Groovy,用Groovy实现Contract Management Web Services的代码生成工具。并将Groovy介绍推广给公司的其他开发人员。
  \item 分析调试已有产品的后台代码(EJB),并根据web services的需求进行修改。
  \end{itemize}
\item 使用Oracle Universal Installer给各个主要产品制作了安装程序。
\item 用Oracle BPM 10g和P6 Web Services实现了Feature Request工作流,并且部署在公司内部,供开发团队使用。
\end{itemize}}
\newline

\cventry{2008.3--2008.12(海外)}{实习程序员}{Primavera System Inc}{}{}{
参与公司产品的开发测试工作,协助其他开发人员开发新特性和发布各个发行版。
\newline\newline
主要成就:
\begin{itemize}
\item 将P6 Web Services产品的web services框架由Axis2改为Apache CXF,深入了解了web services。
\item 用StAX API为P6产品实现了XML导入导出工具。熟悉了P6项目管理软件的业务逻辑。
\item 为P6 Web Services编写Java和C\#的demo程序。
\end{itemize}}


%\section{其他信息}
%\cvline{备注}{
%我目前在美国工作,考虑回国发展,正在寻找合适的机会。\newline
%国内的手机15268861878\newline
%美国的手机1-610-812-9803\newline
%邮件librakevin@gmail.com}

\end{document}
